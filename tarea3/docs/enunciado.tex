\documentstyle[12pt,spanish]{article}
\oddsidemargin 0cm
\textwidth 16cm 
\topmargin -1cm 
\headheight 0cm 
\textheight 22cm
\parindent 3em  
\parskip 3ex    
\begin{document}

{\Large\bf Enunciado Tarea Computaci'on Gr'afica}

Hojeando el {\em Foley}, mirando las fotos de im'agenes gr'aficas, aparece una foto de un
video juego que me result'o muy interesante. Se llama {\bf Block Out} y es algo as'i como
un tetrix en 3D (la foto es el Plate 1.6 del Foley).

La idea del juego consiste en que el jugador debe hacer calzar piezas o figuras de
3D en un peque'no espacio.
El espacio es un paralelep'ipedo de 3$\times$3$\times$10, y las figuras hay que irlas colocando
hacia el fondo de este paralelep'ipedo o caja. 

Las figuras pueden rotarse en torno a los 3 ejes, en 'angulos de 90 grados y -90 grados, y
trasladarlas o moverlas en las direcciones arriba, abajo, derecha, izquierda, 
siempre y cuando no se salga del espacio y no haya otra 
figura que impida el movimiento.

Cuando aparece en pantalla la pieza que hay que colocar al fondo del espacio, 'esta
aparece transparente y una vez que la figura ya haya llegado al fondo quedar'a con un
color dado, y quedar'a fija ah'i, y el jugador ya no la podr'a mover.

An'alogamente como en el tetrix, cuando se ha completado un "muro" (por decirlo de
alguna manera), 'este se elimina y todo lo que estaba delante del muro  se desplazar'a
en 1 unidad hacia el fondo.

Las figuras posibles est'an formadas por combinaciones de la figura elemental que es
el cubo unitario.
Yo ya he dise'nado 11 figuras que son las que aparecen en el video juego.

Esta tarea se puede dividir en las siguientes 2 etapas:

{\large\bf Etapa 1}
\begin{itemize}
    \item Dibujar el espacio (el paralelepipedo de 3$\times$3$\times$10)
    \item Dibujar cada una de las figuras
    \item Permitir rotaciones y traslaciones para cada una de las piezas dentro del
     espacio
\end{itemize}
     Durante los d'ias de paro (5 semanas!) me estuve entreteniendo haciendo esta tarea y me met'i tanto en ella
     que esta etapa est'a lista (creo yo). Los archivos est'an en {\bf anakena} en:
\begin{verbatim}
            ~jlflores/cc52b/tarea2/
\end{verbatim}


{\large\bf Etapa 2}

 Consiste en incorporarle las caracter'isticas del juego al prototipo anterior, esto es:
 
\begin{itemize} 
    \item Generar aleatoriamente una figura dada y dibujarla en el espacio
    \item Permitir rotar (en torno a los 3 ejes) y trasladar (arriba, abajo, derecha, izquierda)
     esa figura.
      El desplazamiento hacia adentro y hacia afuera  no lo maneja el jugador: cada cierto $\Delta$t de
     tiempo la figura actual que se est'a manejando (la cual aparece transparente)
     se desplaza en 1 udd. hacia adentro; o sea, adopta la caracteristica del video
     juego : el jugador no puede hacer retroceder la figura; y as'i, se pueden
     manejar distintos niveles de dificultad variando el valor del $\Delta$t.
    \item Cuando la pieza ya no pueda seguir avanzando hacia adentro porque hay otra
     pieza que se lo impide, o porque ya lleg'o al fondo del paralelep'ipedo, la
     figura adoptar'a un color dado; y as'i, se vuelve a generar otra figura
     aleatoriamente y se sigue la misma idea
    \item Cuando un "muro" se haya completado, 'este se borra y todo lo que estaba delante
     de 'el se corre en 1 udd. hacia adentro.
\end{itemize}
     
     Tambi'en durante el paro estuve trabajando en esta etapa.
     Los archivos est'an en
\begin{verbatim}
            ~jlflores/cc52b/tarea3/
\end{verbatim}        
El juego se ejecuta con {\em tarea3} y funciona as'i:

Las teclas definidas para jugar son A, S, D, Q, W, E para rotar en torno a los 3 ejes en 'angulos de 90 grados y -90
grados y I, J, K, L para trasladarse. Tambi'en est'an definidas las teclas N, P, Esc y la barra de espacio.

\newpage 
{\bf Rotaciones (seg'un la regla de la mano derecha)}
\begin{itemize}   
   \item  A : el pulgar apunta a tu derecha.
    \item S : el pulgar apunta hacia abajo.
    \item D : el pulgar apunta hacia fuera del monitor.

    \item Q : Inverso de A
    \item W : Inverso de S
    \item Z : Inverso de D
\end{itemize}

 
{\bf Traslaciones}  
\begin{itemize}   
   \item  I : arriba
    \item J : izquierda
    \item K : abajo
      \item L : derecha
  
\end{itemize}

 
{\bf Otros botones} 
\begin{itemize}  
    \item P : Pausa
    \item Barra espacio : Tira la figura  r'apido hacia el fondo
    \item N : pasa al siguiente nivel de dificultad. Tengo definido 12 niveles de dificultad.
             Un nivel est'a definido por la dificultad de las figuras que pueden aparecer, y por el $\Delta$t.
\end{itemize}


 


Cualquier observaci'on con respecto a la tarea la acepto sin ning'un problema.

{\bf Jos'e Leonardo Flores Vargas }
  
{\bf jlflores@dcc.uchile.cl}

\end{document}
